Algorithm 1 Domain-Specific Research Idea Generation

1: procedure GenerateResearchIdeas(query, knowledgeGraph, research￾Papers, LLM, numIdeas)
2: concepts ← ExtractKeyConcepts(query)
3: relevantNodes ← FindRelevantNodes(knowledgeGraph, concepts)
4: relatedPapers ← GetRelatedPapers(researchPapers, relevantNodes)
5: ideas ← EmptyList()
6: while Size(ideas) ¡ numIdeas do
7: prompt ← CreatePrompt(concepts, relatedPapers)
8: newIdea ← GenerateIdeaWithLLM(LLM, prompt)
9: if IsNovel(newIdea) AND IsFeasible(newIdea) AND IsRele￾vant(newIdea) then
10: ideas.Add(newIdea)
11: end if
12: end while
13: return ideas
14: end procedure


1 Process Description
This algorithm outlines a method for generating domain-specific research ideas.
It takes a user query, a knowledge graph of the domain, a set of research papers,
a large language model (LLM), and the desired number of ideas as inputs.

Key Steps
1. Extract key concepts from the user’s query
2. Find relevant nodes in the knowledge graph based on these concepts
3. Retrieve related research papers
4. Generate ideas using the LLM, guided by the extracted concepts and
related papers
1
5. Filter ideas based on novelty, feasibility, and relevance
6. Continue generating and filtering until the desired number of ideas is
reached
This process leverages domain knowledge (via the knowledge graph and research papers) and advanced language models to produce relevant and potentially innovative research ideas.